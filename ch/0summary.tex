%%%Summary Abstract
\chapter*{Summary}
\addcontentsline{toc}{chapter}{Summary}
\setheader{Summary}

%Integrating new technologies on training programs for professional athletes have been increases in the last years. Olympic sailing uses the approach of one-boat design hence improvements on the boat are not allowed. Because of this, the scope for many  types of research is on enhancing the strength and other motor skills of athletes rather than strategy methods to help them train. As in many race competitions, the winner is the one that first crosses the end line. Olympic sailing regattas stand for different days and races with distinct configurations for the route and weather conditions. This means the winner had crossed the line not only one time but in most of the races.\par \noindent
%In addition, athletes are not allowed to get any information or feedback from coaches during the race, in consequence, the information of how to course the race is crucial not only for the race but also for its preparation and training. To win, the athletes confront the weather variations and its uncertainty taking different decisions. One of these decisions refers to the sailing direction taken at any start line, the other decision that impacts the results is the number of maneuvers when the boat sails against the wind. \par  % that can be started on the left or right hand side.  , known as a zig-zag pattern

This study develops an algorithm on \acrshort{matlab} to model the path for sailing competitions and optimize the minimal time shaped by a wind of sailing races for the Laser (Olympic class). Furthermore, to identify the effects of time-steps of different size on the resulting times and trajectories. For the validation of the algorithm, this research compares the results between the algorithm using alternative scenarios and the results from the race. One of these scenarios uses the wind measured during the race. The reference race is R1 from the World Cup Series 2018 at Hyères, France. \par  

%Integrating new technologies on training programs for professional athletes have been increases in the last years. Olympic sailing uses the approach of one-boat design hence improvements on the boat are not allowed. Because of this, the scope for many  types of research is on enhancing the strength and other motor skills of athletes rather than strategy methods to help them train. As in many race competitions, the winner is the one that first crosses the end line. Olympic sailing regattas stand for different days and races with distinct configurations for the route and weather conditions. This means the winner had crossed the line not only one time but in most of the races.\par \noindent
%In addition, athletes are not allowed to get any information or feedback from coaches during the race, in consequence, the information of how to course the race is crucial not only for the race but also for its preparation and training. To win, the athletes confront the weather variations and its uncertainty taking different decisions. One of these decisions refers to the sailing direction taken at any start line, the other decision that impacts the results is the number of maneuvers when the boat sails against the wind. \par 

To predict the minimal time path this research reviewed the physical model for sailboats. Sailboats interact with water and air, while the seamanship is the one that controls it to reach any destination. Therefore, the sailboat is a rigid body that can move in a three-dimensional space, however this research only considers the displacement in two dimensions. \par 
Despite the similarities between Laser and yachts, one of the Laser adaptations is %the Laser requires adaptations, one of them is 
the addition of two coefficients %related to its 
to the sail's forces. These coefficients impact the \acrfull{vpp} because it is the result from the interaction between the forces and moments generated by the wind and water when the sailboat is in motion. Moreover, the \acrshort{vpp} shows the direction at which the Laser reaches its maximum velocity at alternative wind's velocities. \par %Another approach to consider is the velocity triangle to know %since it points the direction to sail and 
%the distance at a certain time between two references. \par   %  
Wind is the main source of propulsion for sailboats and the seamanship uses the wind to maximize the velocity of the sailboat to adjust the direction and reach the target. Similarly, this algorithm uses the wind properties to  find the path with the minimal time. The wind model is a forecast of four-dimensional and it describes the wind using a grid to locate the wind's speed over an area. Its resolution is the size of the grid, the distance between points and the time step. \par \noindent 
%The Laser races occurs within a diameter of range [0.8,1.5] nm ([1.482 , 2.778]km ) and it has a duration of 1 hour approx. 
Public information about wind forecast has a resolution distant from the characteristics of the race. The wind model used here is a \acrfull{wrf} with a grid resolution of 1km, a time step of 10 minutes and an initial height of 7.5 meters over the area between 41.66\degree N, 4.75\degree E and 44.45\degree N,  7.25\degree E coordinates. %To translate the wind velocities provided by the model to the \acrfull{ce} of the sail the algorithm uses a power relation between velocities and heights. 
While the measured data provided about the wind is in a tabular layout with a sample rate of 20 Hz. at 5 locations around the area of the event. \par 
This algorithm %developed here 
discretize the area of the race %and its surroundings 
to break the problem into multiple stages all connected and to have grid points for the wind model. Because of this, the technique used is dynamic programming with a heading angle direction given by the \acrshort{vpp}. \par 

The scenarios used within this algorithm include step time variations of the \acrshort{wrf} wind model and the wind measurements from the race besides the  values of the wind properties without spatial variation along the race area. Thus the scenarios to simulate the optimal time path are constant and uniform wind, wind field with a spatial resolution of 1km and a time step of 1 hour and 0.5 hours, the \acrshort{wrf} and the wind measurements from the race. \par 
In conclusion, the leg-times, start points over the race-lines and directions of the paths compose the minimal time path. Two of these elements were predicted similarly as the winners using the \acrshort{wrf} wind model with an error in the race-time for less than 5\%. However, the direction of the paths was not predicted accurately for the upwind-mode legs. This was equivalent to the winner using the constant and uniform wind scenario where the percentage error of the race-time respect to the winner is about 7.82\%. Although here, the direction of leg 2 is not similar to the winners. \par \noindent
The grid resolution of the wind model is important for the minimal time path estimation. Moreover, the racecourse must be within the area of the wind model to avoid the extrapolation of the wind properties %. Because extrapolated values of the wind properties rise 
and percentage errors (\%) larger than 25\%. respect to the winner for leg-times sailed under upwind-mode. When the race area is not within the wind model area, the average values of the wind properties produce trajectories with smaller errors in time and shape. \par 
This research proposes to review the location of the \acrshort{ce} because at shorter distances from the sea level the current and height of the waves could generate frictional forces insignificant at higher distances. Another recommendation is to examine why for downwind-mode legs the changes in the direction do not affect the time as in the upwind mode. Finally, review the influence of the position of the sail-man using a three-dimensional model with the presence of waves and current during downwind and direct wind modes. \par 
 